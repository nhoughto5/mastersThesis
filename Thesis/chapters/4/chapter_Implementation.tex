\startchapter{Software Implementation}
\label{chapter:implementation}
\section{Introduction}
The \Name was build to be a stand-alone application.
In order to allow it to be a powerful yet simple application the technologies used were selected with the idea of portability in mind.

\section{Technologies Used}
\subsection{\Xilinx}
\Xilinx is one of the two largest manufacturers of \acrshort{FPGA}s. 
Their devices are considerably more popular than their competitors.
The configuration of devices employs a well known series of steps referred to as the \Xilinx 'tool-chain'~\cite{xilnxDevManual}.
The 'tool-chain' not only compiles user designs and constraints into the configuration \gls{Bitstream} but performs a series of complex operations to optimize designs and effectively implement them on any \Xilinx model of the user's choosing.
\begin{enumerate}
	\item \textbf{\gls{sch2hdl}}: \glsdesc{sch2hdl}
	\item \textbf{\gls{XST}}: \glsdesc{XST}
	\item \textbf{\gls{MAP}}: \glsdesc{MAP}
	\item \textbf{\gls{PAR}}: \glsdesc{PAR}
	\item \textbf{\gls{ngdbuild}}: \glsdesc{ngdbuild}
	\item \textbf{\gls{trce}}: \glsdesc{trce}
	\item \textbf{\gls{Bitgen}}: \glsdesc{Bitgen}
\end{enumerate}
Each step produces a series of files that are used for a variety of purposes. 
Often the resultant files are used by the subsequent step but some are intended for user information.
The \gls{ngdbuild} tool generates what is known as the netlist for the design.
The netlist is a description of the connectivity of the circuit implemented on the device. 
The generated netlist is in a non human-readable format in an 'ndc' file.
Fortunately, \Xilinx provides an additional tool called \acrshort{xdl2ndc} which allows the conversion to the human-readable \acrfull{XDL}.
\subsubsection{\acrfull{XDL}}
The \acrfull{XDL} is a human-readable ASCII format; though it is not actively part of the 'tool-chain' it is considered a native netlist format for describing and representing \acrshort{FPGA} designs~\cite{xdlTutorial}. 
A part is a human-defined component which is to be implemented.
Netlists either contain or refer to descriptions of the parts used and where in the device they are implemented.
When a part is used it is called an 'instance'; thus each 'instance' has a definition, sometimes referred to as a master.

An \acrshort{XDL} file contains two sections, the instance placement and configuration section, and the net routing section. 
The placement and configuration section provides a list of every instance in a design. 
Their descriptions include all of the configuration details required for their implementation on a component (power settings, logic, timing configurations... etc).
In the \Xilinx jargon, a net refers to any electrical path between two components; more specifically, a net describes a communication channel.
The gate-array of a \Xilinx device is composed of a grid of wires.
These wires can be fused together by \acrfull{PIP} to make a useful connection between two components, thus creating a net.
The output-pin of a net receives the signal from the transmitting component while the input-pin delivers the signal to the receiver.
The \acrshort{PIP}s in-between dictate the path the signal takes between the two components.
The net routing section of the \acrshort{XDL} file describes every path in the design.
Combined, these two sections completely describe a design and how it is implemented.
\subsubsection{XDLRC}
The \acrshort{XDL} file describes the design implemented on a device.
From this a lot of information regarding the composition of a \Xilinx device can be learned but it does not provide the entire description; only what has been used by the design.
The \acrshort{xdl2ndc} command line tool provides an option to generate a resource report file referred to as an XDLRC file. 
A XDLRC file describes the entire architecture of a \Xilinx \acrshort{FPGA}.
\ConditionSize
\begin{lstlisting}[label={lst:xdlrc}, language=Python, caption={A hierarchical XDLRC resource description of a Spartan 6 FPGA consisting of a header, a tile section, and a trailing device summary~\cite{xdlTutorial}}]
#header
(xdl_resource_report v0.2 xc6slx16csg324-3 spartan6
# Device Level Dimensions
(tiles 73 62
...
	#Configurable logic block with two slices
	(tile 4 6 CLEXL_X1Y61 CLEXL 2
		(primitive_site SLICE_X0Y61 SLICEL internal 45
			(pinwire A1 input L_A1)
		...
		(primitive_site SLICE_X1Y61 SLICEX internal 43
		...
		(pinwire D output XX_D)
	...)
	# Interconnect tile
	(tile 4 5 INT_X1Y61 INT 1
	...
		(wire EE2B0 2
			(conn CLEXM_X2Y61 CLEXM_EE2M0)
			(conn INT_BRAM_X3Y61 EE2E0)
		...
		# Programmable Interconnect Points
		(pip INT_X1Y61 EE2E0 -> EE2B0)
		(pip INT_X1Y61 EE4E0 -> EE2B0)
		(pip INT_X1Y61 EL1E_S0 -> LOGICIN_B9)
	...)
# summary
(summary tiles=4526 sites=5378 sitedefs=46 numpins=157962 numpips=5782505))
\end{lstlisting}
\normalsize

Listing~\ref{lst:xdlrc} provides and example of the XDLRC format.
The report begins with a header describing the device.
It then reports that the overall architecture contains a matrix of tiles 73-wide and 62-tall.
Further down we a configurable logic block at coordinate (4,6) and an interconnect tile at coordinates (4,5) are shown.
\subsection{RapidSmith}
\subsubsection{Database Files}
\subsubsection{}
\subsection{Java Swing}
\subsection{Java}
\section{Design}
\section{\acrfull{UI}}