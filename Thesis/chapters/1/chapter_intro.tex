\startfirstchapter{Introduction}
\label{chapter:introduction}
The term \textit{Trojan Horse} or \textit{Trojan} has become a modern metaphor for a deception where by an unsuspecting victim welcomes a foe into an otherwise safe environment~\cite{searchForTrojanWar}.
Though modern civilization rarely has need for large walls we are similarly surrounded.
Not by stone and mortar but by the technology we so heavily rely on.
These days it is more common to come across a piece of equipment with some form of computer in it than without.
They provide us entertainment, education, security, monitor our health, grow our food and more.
Our reliance make us susceptible to their compromise.
Since the dawn of the computer we have dealt with software threats.
We are almost as good at protecting ourselves against them as attackers are at making them.
In recent years a new incarnation of electronic danger has emerged; in hardware.
In this new arena of attack and defend those who seek to defend are far behind.

\section{Motivation}
In the summer of 2007 an Israeli military action referred to as Operation Orchard commenced.
A group of F-15I Ra'am fighter jets from the Israeli Air Force 69th Squadron took off to attack a suspected nuclear reactor in neighboring Syria~\cite{stoppingHTsIEEESpectrum}.
In the flight path was a Syrian radar station which boasted ``state-of-the-art'' aircraft detection and neutralization technology. 
The Israeli war planes were able to approach and destroy the installation undetected.
Though never proven it is commonly accepted that the detection mechanism was deactivated by a back-door circuit inserted into the radar system.
In 2011 over 1300 cases of modified \acrshort{IC}s were reported to the \acrfull{ERAI}; the occurrence of such cases has since been going up~\cite{counterfeitIEEESpectrum}.
\acrfull{ICs} are a large part of modern life yet it is easy to forget that they drive virtually every piece of technology used today.
Ensuring that their \acrshort{IC}s run our devices as expected is vital in the digital era.
Since their discovery there has been concerted effort to detect these modifications but the innate complexity of \acrshort{IC}s makes it difficult~\cite{hardwareTrojanSurvey2015}.
These modifications are commonly known as hardware trojans.

One of the most commonly attempted trojan detection techniques is to apply a series of inputs and look to see if the outputs are as expected.
This is known as Functional testing or Test Vectoring \cite{kSubset,monteCarloTestPattern,towardsDetectionMethodology}.
This method has been shown to provide some success but is commonly beaten by attackers.
Another common strategy is to search for side effects of modified circuits; these methods are referred to as ``side-channel'' analysis methods.
Such side effects include, changes to power consumption, temperature differences, radiation differences and more~\cite{sideChannelObfuscation, postLayout, controllableSleepTransistors, pcaAlgorithm}.
Again, these methods provide varying success but face a lot of difficulty.
Trojans and detection methods alike are new.
The work in this field has only seen progress in the last few decades.
The results of this work are still primarily buried in academic publications in various locations, in various formats.
The discovery of what has already been achieved and what still needs to be done in this field is arduous 

\acrshort{IC} designs for \acrfull{FPGAs} are made using a software language.
The design is then converted to a binary file called a configuration \gls{Bitstream} which is then downloaded onto the device; this process is known as synthesizing the design.
There have been many attempts to develop mechanisms and techniques to determine whether a malicious user has tampered with the design via test vectoring or side-channel analysis.
As of yet there has been little effort to directly analyze the configuration \gls{Bitstream}.


\section{Contributions}
The goal of this work is two-fold.
First, to prove that analysis of the configuration \gls{Bitstream} is the most powerful means of detecting trojans in \acrshort{FPGA}s.
To do so, an application was developed which parses and analyzes the \gls{Bitstream}, detects the presence of trojans and provides an astute description of the modification.
Secondly, to provide a centralized place for researchers to review and analyze discovered trojans.
A website was developed which provides the automation of useful analysis techniques as well as a database, yet to be populated, of all known trojans and detection methods. 

The contributions of this work can be said to be:
\begin{enumerate}
	\item Proof that though \acrshort{FPGA} manufacturers hide the intimate details of the configuration \gls{Bitstream} it is the most viable means of trojan detection and analysis. 
	\item Proof of concept for a new method of detecting and analyzing trojans.
	\item Develop a software application which automates the described methodology.
	\item Develop a database structure to house all known trojans and detection methods
	\item A web site to interact with and access the trojan/detection method database.
\end{enumerate}
\section{Organization}
This enclosing section presents a map of this work and a short description of each chapter.
Chapter 2 provides some background information on trojans themselves and introduces the descriptive taxonomy used.
Chapter 3 introduces the new detection and analysis methodology.
It gives a brief overview of \acrshort{FPGA} architecture and configuration, describes the analysis process of the \gls{Bitstream}, introduces \textit{Component Mapping}, and discusses how the taxonomic attributes are extracted.
Chapter 4 presents an online 
%Chapter 4 discusses the software implementation which has been named \Name.
It discusses the technologies used and why they were chosen, presents the \acrlong{UI} and discusses its operating procedure.
Chapter 5 presents the results of experimentation done using three \acrshort{FPGA}-trojan benchmarks
These benchmarks were specifically chosen to demonstrate that the application works as expected and is easy to use.
Finally, chapter 6 concludes this work and restates its contributions.