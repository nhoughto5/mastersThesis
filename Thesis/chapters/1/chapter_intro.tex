\startfirstchapter{Introduction}
\label{chapter:introduction}
The term \textit{Trojan Horse} or \textit{Trojan} has become a modern metaphor for a deception where by an unsuspecting victim welcomes a foe into an otherwise safe environment.~\cite{searchForTrojanWar}
%Grand tales have been told of large wooden horses though scholars believe the true trojan was far more practical.\cite{searchForTrojanWar}
%Covered battering rams were commonly used at the time for gaining access to fortified areas. 
%Such a device is thought to be the real Trojan horse.
Though modern civilization rarely has need for large walls we are similarly surrounded.
Not by stone and mortar but by the technology we so heavily rely on.
These days it is more common to come across a piece of equipment with some form of computer in it than without.
They provide us entertainment, education, security, monitor our health, grow our food and more.
Our reliance make us susceptible to their compromise.
Since the dawn of the computer we have dealt with software threats; we are almost as good as protecting ourselves against them as they are at attacking us.
In recent years a new incarnation of danger has emerged; in hardware.
In this new arena of attack and defend those who seek to defend are far behind.

\section{Motivation}
In the summer of 2007 an Israeli military action referred to as Operation Orchard commenced.
A group of F-15I Ra'am fighter jets from the Israeli Air Force 69th Squadron took off to attack a suspected nuclear reactor in neighboring Syria.
In the flight path was a Syrian radar station which boasted 'state-of-the-art' aircraft detection and neutralization technology. 
The Israeli war planes were able to approach and destroy the installation undetected.
Though never proven it is commonly accepted that the detection mechanism was deactivated by a back-door circuit inserted into the radar system.~\cite{stoppingHTsIEEESpectrum}
In 2011 over 1300 cases were reported to the \acrfull{ERAI} of modified \acrshort{IC}s and the occurrence has been going up.~\cite{counterfeitIEEESpectrum}
\acrfull{ICs} are a large part of modern life yet it is easy to forget that they drive virtually every piece of technology used today.
Ensuring their \acrshort{IC}s run our devices as expected is vital in the digital era.
Since their discovery there has been concerted effort to detect hardware trojans but the innate complexity of \acrshort{IC}s makes it difficult.~\cite{hardwareTrojanSurvey2015}

\section{Contributions}
\section{Organization}


%The main goal of this document was to set up competently the Latex framework for a thesis document at UVic, with all the assorted details of front pages, numbering and so on. In order to present such a template to other users, the content has been filled from the guidelines of an experienced graduate advisor and supervisor on how to write a thesis in the first place.
%
%The Latex framework given in these files should work in the Windows, Mac and Unix environments. In Windows, the development environment is MikTex, available as a free download from the web page at \textit{miktex.org}. Miktex includes all the tools for \LaTeX, without a convenient editor. I have been using
%\textit{WinEdt} which is still the best as far as working in a well integrated manner with MikTex, even if one has to pay a small price. \textit{WinEdt} is available from \textit{winedt.com}. The corresponding package for the Mac is MacTex, available from \textit{www.tug.org$\backslash$mactex}.
%Please note that in the actual file with the \LaTeX code there are extra local comments referring to small differences between platforms. This is especially crucial for the insertion of figures of any type (see also Chapter 3).
%
%The package \textit{BibTex} is used for the automatic bibliography and it is incorporated in the MikTex package. The style of bibliography exemplified is this document is the "plain" one,
%most often used in science theses. This is shown
%by the entry \textit{plain} in the \textit{bibliographystyle}
%command which can be found in the top file, namely
%\textit{mainthesisUVIC.tex}. Substitute the
%appropriate bibliography style for your needs by finding
%the correct parameter. In order to help you there is a PDF
%file entitled: "InformationOnBibliographyStyles.pdf" in the
%same folder as the top file \textit{mainthesisUVIC.tex}.
%
%The title pages are correctly formatted according to the guidelines. However the superset is given here as a template for a Ph.D. dissertation and adjustments must be made for a Master's thesis accordingly.
%
%When presented with the task of having to write a large document reporting on the work done for the thesis portion of a graduate degree most people have anxiety attacks. It seems that the research work has been done, probably with great competence, yet one feels totally lost on how to approach the writing part - unless of course one is a writer to start with. While I am not at all the expert, I have experience both in writing documents and in supervising many graduate students. I have developed a simple enough structure which guides me and my students most of the times fairly well. We then customize it for different audiences and situations. The content of this pseudo mini thesis summarizes the structure I use.
%
%This document, however, is \textit{not at all} a Latex manual. There are a few examples sprinkled here and there of various commands. However no explanations or full examples are given for how to use Latex itself. That is most certainly left to the user!
%
%Starting at the very beginning, my first step is to set up the formatting framework before the content itself, just like the frame of a building is the sustaining structure. By having the content only in bullet points which can be moved and changed at will I can have a whole document which will always compile and be ready for presentation. I ask my student to prepare mock chapters which contains only these bulleted lists to be used for discussion. The bulleted list I had for this chapter is as follows:
%\begin{enumerate}
%        \item State the dual purpose of this document (just mention briefly);
%        \item State what this document is not (just mention);
%        \item Describe what I think should go in the Introduction in general;
%        \item Describe what I think should not go in the "Introduction" chapter in general;
%        \item Describe the structure of the document.
%\end{enumerate}
%
%When the writing for a chapter is particularly long, it might be better to have some, if not all, sections in separate files. It makes it a lot easier to find possible errors in Latex as well, since the command to input a file for a separate section can always be commented out. When looking at the \textit{.tex} file itself for this introduction chapter you will notice that there are 2 sections, namely "How to Start an Introduction" and "Is a Review of All previous Work Necessary Here?" which are included as separate files, immediately following this paragraph.
%\section{How to Start an Introduction}
%
%
%What is the difference between an "Introduction" and the "Background" for a thesis? Should they be together? What about the review of the work from other people?
%
%These are important questions which must be discussed between a student and the supervisor.
%
%My view is that the introduction should be exactly that: a short introduction, not the history of the problem, its content and all its solutions to date. One major ingredient \textit{must} be a marketing angle, such that the reader becomes deeply motivated to continue on with the rest of the document. When I referee an article I usually read the abstract, the introduction and the conclusion. At that point I expect to be able to state to someone else what the work is about, what seems to be the new advancement and how interesting I think it will be to read the rest.
%
%My main statement about the introduction is:
%\textit{"keep it short and write it last".} The main features should be as follows:
%\begin{enumerate}
%	\item {3-4 pages at most;}
%	\item {Start with a VERY short statement of the problem (2-3 sentences) - the problem should be stated, not described, as there will be a whole chapter for that;}
%	\item {State why the problem is important, its impact, how well it has been studied recently, its application (3 sentences) - this should be again a brief motivation, leaving a full impact description to later in the document;}
%	\item {Give a sketch of the new approach - there will be a whole chapter with all the details, now just impress the reader about what is the new approach, just as you would do if your boss asked you at work during an elevator ride;}
%	\item {Sketch the main new ideas of the new approach - again briefly, just get the reader interested;}
%	\item {Give a short statement regarding the results, nothing too elaborate, but certainly you should blow your horn and make sure that the reader is intrigued;}
%	\item {Interspersed in all the writing above do not forget the marketing angle, trying to suggest forcefully why the reader should keep reading;}
%	\item {Give an outline of what is to come in the organization of the thesis overall - you will find one below for this document.}
%	
%\end{enumerate}
%
%Finally the strong suggestion is write the introduction chapter last. It will be faster, you will know what to say as the rest is already there, and the abstract, introduction and conclusion will be a mirror and complement of each other. You may well ask where to start writing your thesis. My view is included in the organization below.
%\section{Is a Review of All Previous Work Necessary Here?}
%
%Many people like to place in the introductory chapter a review of the work from everybody else on the same problem. I find this utterly boring and counter-productive. If I am an expert in the field I probably know all this and could even write it better, so the last thing I want to read at the beginning is a history of research. If I am not an expert in the field, I would prefer to read up all the details of the problem itself and understand its context before I can even take in any ideas of what others have done.
%
%Secondly, often when talking about the work of others one includes a bit judgment on it. This may be necessary, as the pivot of the new work may indeed be that there was an open problem left unsolved by the other researchers. Yet it is hard not to sound negative, to describe the work of others stating that their solution did not include some important part without denigrating. If one of those researchers is a reader and perhaps an examiner, they do not want to start reading by being told that their own work is inadequate.
%
%I suggest a general summary of no more than 3-4 paragraphs of the work in the area with references, just to give the impact of the importance of the problems to be solved. Use then a "just-in-time" approach, where the relevant work of another researcher is described when needed in a particular step of the exposition of the problem or of the new solution. I normally place a much more extensive review of the work of others after I have presented my solution and the results of my experimental or formal proof work, so that I can analyze and compare effectively without boring repetitions and without writing a literature review on a subject. Many good articles in journals follow this scheme and most grant organization ask for a literature review to be included at the end of an application. First they want to be impressed with the proposal for new work!
%
%A similar logic should be applied to background knowledge and definitions. This is especially true in a scientific thesis where often one find a whole set of mathematical definitions lumped together in chapter 1, yet not used until chapter 6, by which time the reader has completely forgotten it and needs to shuffle back with irritation. Use again a "just-in-time" approach and give definitions and explanations locally, the first time they are needed.
%
%%\pagebreak
%\section{My Claims}
%Something must be new in this work, no matter how small, since you are getting a graduate degree for it! Tell me about it clearly and succintly right now, just as you did in the abstract. Make an impact here. How about something like the following box:
%
%I make \textit{four} claims which
%my dissertation validates:
%\\
%
%\framebox{%
%\parbox{5in}{
%	My new algorithm to solve the problem of doing nothing include these important new features whose practical applicability can be proved both formally and empirically:
%	\begin{enumerate}
%	\item first feature;
%	\item second feature;
%	\item everything is much easier to understand, and therefore, easier to implement correctly.
%	\end{enumerate}
%}}
%\\
%
%\noindent Claim 1 and claim 2 are \textit{quantitative} - they will be proven by experiment.
%
%\noindent Claim 3 is \textit{qualitative} - they will be demonstrated by argument.
%
%\subsection{The Importance of My Claims}
%
%Some very important positive consequences
%arise from the validation of the above claims.
%It is these consequences that comprise a significant
%positive contribution to research in the field
%of whatever the field is.
%\\
%
%\noindent Claim 1 implies that:
%\begin{enumerate}
%\item{Something profound which applies to:
%	\begin{itemize}
%	\item {something excellent;}
%	\item {something important.}
%	\end{itemize}}
%\item{Something else just as profound.}
%\end{enumerate}
%
%\noindent Claim 2 implies that:
%\begin{itemize}
%\item{Repeat as above if necessary and useful.}
%\end{itemize}
%
%\noindent The consequence of claim 3 is that:
%\begin{itemize}
%\item{There must be something good coming out of all this work!}
%\end{itemize}
%
%\section{Agenda}
%
%This section provides a map of the dissertation
%to show the reader where and how it validates
%the claims previously made. Here is where I am also presenting my own style of organization which may be totally different from what your supervisor thinks. However, trust me, this is a good solid beginning for a structure. Your supervisor may ask you to change it, but will still appreciate what you have! For each of the chapters below I also give a short summary of what the main focus should be and then I expand on it  a bit within the chapter itself.
%
%\begin{description}
%\item[\textbf{Chapter 1}] contains a statement of
%the claims which will be proved by this dissertation followed by an overview of the structure of the document itself.
%\item[\textbf{Chapter 2}] describes in details the open problem which is to be tackled together with its context, its impact and the overall motivation for the research overall.
%\item[\textbf{Chapter 3}] gives the new research, its methodology, the algorithms involved, the new solution, the new work done. Formal proofs and arguments are made here. This is the first of the two contributions expected in a thesis for a graduate degree.
%\item[\textbf{Chapter 4}] is where the experiments and the methodology for them is fully described. The first part includes all details of how the empirical side of the research has been conducted. Note that not every thesis has this empirical portion.
%\item[\textbf{Chapter 5}] includes the evaluation of the data presented above and the comparisons with the work of others, to show how much better the new approach is. This is the second of the two contributions expected in a thesis for a graduate degree. Note that this part could be consolidated into the chapter above.
%\item[\textbf{Chapter 6}] contains a restatement of the claims and results of the dissertation. It also enumerates avenues of future work for further development of the concept and its applications.
%\end{description}
%
%The list above is not complete. Chapter 3 actually includes a lot more, as I could not resist placing in it a few \LaTeX examples to help you along. This document is not a primer for \LaTeX, but there is no harm done in giving a little help.
