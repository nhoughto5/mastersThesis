\newpage
\TOCadd{Abstract}

\noindent \textbf{Supervisory Committee}
\tpbreak
\panel

\begin{center}
\textbf{ABSTRACT}
\end{center}

Electronics have become such a staple in modern life that we are just as affected by their vulnerabilities as they are.
Ensuring that the processors that control them are secure is paramount to our intellectual safety, our financial safety, our privacy, and even our personal safety.
The market for integrated circuits is steadily being consumed by a reconfigurable type of processor known as a field-programmable gate-array (FPGA).
The very features that make this type of device so successful also make them susceptible to attack.
FPGAs are reconfigured by software; this makes it easy for attackers to make modification.
Such modifications are known as hardware trojans.
There have been many techniques and strategies to ensure that these devices are free from trojans but few have taken advantage of the central feature of these devices.
The configuration Bitstream is the binary file which programs these devices.
By extracting and analyzing it, a much more accurate and efficient means of detecting trojans can be achieved.
This discussion presents a new methodology for exploiting the power of the configuration Bitstream to detect and described hardware trojans.
A software application is developed that automates this methodology.
%This document is a possible Latex framework for a thesis or dissertation at UVic. It should work in the Windows, Mac and Unix environments. The content is based on the experience of one supervisor and graduate advisor. It explains the organization that can help write a thesis, especially in a scientific environment where the research contains experimental results as well. There is no claim that this is the \textit{best} or \textit{only} way to structure such a document. Yet in the majority of cases it serves extremely well as a sound basis which can be customized according to the requirements of the members of the supervisory committee and the topic of  research. Additionally some examples on using \LaTeX are included as a bonus for beginners.
