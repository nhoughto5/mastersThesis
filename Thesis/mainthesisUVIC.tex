\documentclass[12pt,oneside]{book}
\include{macros/style}
\include{macros/use_packages}
\usepackage{cite}
\usepackage[acronym]{glossaries}
\usepackage{nomencl}
\usepackage{array}
\usepackage{mathtools}
\usepackage{multicol}
\usepackage[flushleft]{threeparttable}
\usepackage{amsmath}
\usepackage{listings}
\usepackage{color}
\usepackage{graphicx}
\usepackage{caption}
\usepackage{subcaption}
\usepackage{siunitx}
%\loadglsentries{Utils/Glossary}
\makenomenclature
\renewcommand{\nomname}{List of Symbols}
\newcommand{\Xilinx}{\textit{Xilinx}~}
\newcommand{\ConditionSize}{\footnotesize}
\newcommand{\Name}{Automated Hardware Trojan System~}
\newcommand{\Swing}{\textit{Swing}~}
\newcommand{\RapidSmith}{\textit{RapidSmith}~}
\DeclarePairedDelimiter\ceil{\lceil}{\rceil}
\DeclarePairedDelimiter\floor{\lfloor}{\rfloor}
\definecolor{dkgreen}{rgb}{0.1,0.5,0}
\definecolor{gray}{rgb}{0.5,0.5,0.5}
\definecolor{mauve}{rgb}{0.58,0,0.82}

\lstset{frame=tb,
	aboveskip=1mm,
	belowskip=1mm,
	showstringspaces=false,
	columns=flexible,
	basicstyle={\small\ttfamily},
	numbers=none,
	numberstyle=\tiny\color{gray},
	keywordstyle=\color{blue},
	commentstyle=\color{dkgreen},
	stringstyle=\color{mauve},
	breaklines=true,
	breakatwhitespace=true,
	tabsize=3
}
\newenvironment{conditions}
{\par\vspace{\abovedisplayskip}\noindent\begin{tabular}{>{$}l<{$} @{${}={}$} l}}
	{\end{tabular}\par\vspace{\belowdisplayskip}}

%%%%%%%%%%%%%%%%%%%%%%%%%%%%%%%%%%%%%%%%%%%%%%%%%%%%%%%%%
%To compile glossary and acronyms run following sequence under Tools
%Commands -> pdfLatex
%User -> MakeGlossaries
%Commands -> Make Index
%Commands -> pdfLatex
%Build
%%%%%%%%%%%%%%%%%%%%%%%%Acronyms%%%%%%%%%%%%%%%%%%%%%%%%%
\newacronym{FPGA}{FPGA}{Field Programmable Gate-Array}
\newacronym{FPGAs}{FPGA}{Field Programmable Gate-Arrays}
\newacronym{CL}{CL}{Configurable Logic}
\newacronym{ASIC}{ASIC}{Application Specific Integrated Circuit}
\newacronym{HDL}{HDL}{Hardware Description Language}
\newacronym{ISE}{ISE}{Integrated Synthesis Environment}
\newacronym{IOB}{IOB}{Input-Output Block}
\newacronym{CLB}{CLB}{Configurable Logic Block}
\newacronym{IOBs}{IOBs}{Input-Output Blocks}
\newacronym{CLBs}{CLBs}{Configurable Logic Blocks}
\newacronym{IOI}{IOI}{Input-Output Interconnect}
\newacronym{IOIs}{IOIs}{Input-Output Interconnects}
\newacronym{LUT}{LUT}{Look-up Table}
\newacronym{IR}{IR}{Interconnect Resources}
\newacronym{SM}{SM}{Switching Matrix}
\newacronym{PIP}{PIP}{Programmable-Interconnect-Points}
\newacronym{XDL}{XDL}{\Xilinx Design Language}
\newacronym{RDB}{RDB}{Readback File}
\newacronym{MSD}{MSD}{The mask file used to hide undesirable configuration information}
\newacronym{DFF}{DFF}{D Flip-Flop}
\newacronym{TCL}{TCL}{Tool Command Line}
\newacronym{JTAG}{JTAG}{Joint Test Action Group}
\newacronym{SRL}{SRL}{Shift Right Left Module}
\newacronym{RAM}{RAM}{Random Access Memory}
\newacronym{LL}{LL}{Logic Allocation File}
\newacronym{RBB}{RBB}{Repeatable Building Blocks}
\newacronym{RBA}{RBA}{Readback ASCII}
\newacronym{RTL}{RTL}{Registry Transfer Level}
\newacronym{TV}{TV}{Test Vectors}
\newacronym{XOR}{XOR}{Exclusive OR}
\newacronym{XNOR}{XNOR}{Exclusive NOR}
\newacronym{UUT}{UUT}{Unit Under Test}
\newacronym{PROM}{PROM}{Programmable Read-Only Memory}
\newacronym{GUI}{GUI}{Graphical User-Interface}
\newacronym{EDIF}{EDIF}{Electronic Data Interchange Format}
\newacronym{NCF}{NCF}{Netlist Constraints File}
\newacronym{UI}{UI}{User-Interface}
\newacronym{UIs}{UI}{User-Interfaces}
\newacronym{DCM}{DCM}{Digital Clock Managers}
\newacronym{ROM}{ROM}{Read-Only Memory}
\newacronym{BEL}{BEL}{Basic Element}
\newacronym{UCF}{UCF}{User Constraint File}
\newacronym{API}{API}{Application Programming Interface}
\newacronym{APIs}{API}{Application Programming Interfaces}
\newacronym{CAD}{CAD}{Computer Aided Design}
\newacronym{IO}{IO}{Input-Output}
\newacronym{BRAM}{BRAM}{Block RAM}
\newacronym{SRAM}{SRAM}{Static RAM}
\newacronym{CMD}{CMD}{Command Register}
\newacronym{WCFG}{WCFG}{CMD Write Packet Data}
\newacronym{MSK}{MSK}{Readback Mask File}
\newacronym{appName}{\textit{F-TRAP}}{\acrshort{FPGA} Trojan Recognition and Parsing}
\newacronym{FLR}{FLR}{Frame Length Register}
\newacronym{IC}{IC}{Integrated Circuit}
\newacronym{ICs}{IC}{Integrated Circuits}
\newacronym{PLD}{PLD}{Programmable Logic Device}
\newacronym{PLDs}{PLD}{Programmable Logic Devices}
\newacronym{ERAI}{ERAI}{Electronic Resellers Association International}
\newacronym{BA}{BA}{Block Address}
\newacronym{GCLK}{GCLK}{Global Clock}
\newacronym{PC}{PC}{Personal Computer}
%%%%%%%%%%%%%%%%%%%%%%%%Glossary Entries%%%%%%%%%%%%%%%%%
\newglossaryentry{log}{name={log},description={A log file generated by XST}}
\newglossaryentry{Bitstream}{name={Bitstream},description={The sequence of one and zeroes which makes up the configuration data}}
\newglossaryentry{Bit}{name={Bit},description={The binary file containing the configuration data for the implementation}}
\newglossaryentry{Vivado}{name={Vivado},description={The Integrated Development Environment Used Exclusively for 7-series FPGAs}}
\newglossaryentry{Xilinx}{name={\Xilinx},description={The largest producer of FPGAs used exclusively for this proposal}}
\newglossaryentry{Readback}{name={Readback},description={A feature in all \Xilinx FPGAs where by the current state of the \gls{gateArray} is stored in the configuration memory to be read by the user via the JTAG port}}
\newglossaryentry{SelectMap}{name={SelectMap},description={A \Xilinx device configuration mode which allows for the programming of multiple devices in parallel}}
\newglossaryentry{golden}{name={Golden},description={The clean, untampered version of the synthesis files generated by XST}}
\newglossaryentry{target}{name={Target},description={The generated files from devices received from the third-party fabrication house}}
\newglossaryentry{Library}{name={\textit{Library}},description={The list of absolute addresses for all components in a \gls{gateArray}}}
\newglossaryentry{NGC}{name={NGC},description={The NGC file is a netlist that contains both logical design data and constraints. The NGC file takes the place of both \acrfull{EDIF} and \acrfull{NCF} files}}
\newglossaryentry{Plan Ahead}{name={Plan Ahead},description={Software provided by \Xilinx for RTL to bitstream design flow management}}
\newglossaryentry{RAM16}{name={RAM16},description={\acrshort{LUT} used as a 16x1 memory unit}}
\newglossaryentry{SRL16}{name={SRL16},description={\acrshort{LUT} used as a 16-bit shift register}}
\newglossaryentry{SLICEM}{name={SLICEM},description={A component of a \acrshort{CLB} in the Spartan-3E family of \acrshort{FPGA}s capable of both logic and memory functions}}
\newglossaryentry{SLICEL}{name={SLICEL},description={A component of a \acrshort{CLB} in the Spartan-3E family of \acrshort{FPGA}s capable of only logic functions}}
\newglossaryentry{sch2hdl}{name={sch2hdl},description={used to convert schematic designs to \gls{HDL}. This stage is optional as many choose to develop directly in \acrshort{HDL}}}
\newglossaryentry{XST}{name={XST},description={\gls{Xilinx} Synthesis Tools, a package of which parse, optimize and compile the \acrshort{HDL} code.}}
\newglossaryentry{ngdbuild}{name={ngdbuild},description={A tool used to build a NETLIST.}}
\newglossaryentry{MAP}{name={MAP},description={A \gls{Xilinx} tool used to calculate the optimal routing for the finalized design.}}
\newglossaryentry{PAR}{name={PAR},description={used to place and route the design in the specific \Xilinx device.}}
\newglossaryentry{trce}{name={trce},description={used to perform timing and performance analysis.}}
\newglossaryentry{Bitgen}{name={Bitgen},description={The final stage used to generate the configuration \gls{Bitstream} which will configure the device.}}
\newglossaryentry{xdl2ndc}{name={xdl2ndc},description={A command line tool used to convert between the non human-readable ndc netlist format and the human readable \acrshort{XDL} format.}}
\newglossaryentry{thread}{name={thread},description={The smallest sequence of programmed instructions that can be managed independently by a scheduler, which is typically a part of the operating system.}}
\newglossaryentry{Group}{name={Group},description={A collection of columns of 1-bit wide components in an FPGA}}
\newglossaryentry{FragementMatrix}{name={Fragement Matrix},description={A conceptual interpretation of the architecture of an FPGA where by it is imagined as a matrix of 1-bit configurable components}}
\newglossaryentry{QT}{name={QT},description={ a cross-platform application framework that is widely used for developing application software that can be run on various software and hardware platforms. It provides easy to use \acrshort{UI} design tools and features used for this project.~\cite{QT}}}
\newglossaryentry{Boost}{name={Boost},description={ a set of libraries for the C++ programming language that provide support for tasks and structures such as linear algebra, pseudorandom number generation, multithreading, image processing, regular expressions, and unit testing.~\cite{boostLibrary}}}
\newglossaryentry{winAPI}{name={Windows \gls{API}},description={Windows API or WinAPI is a core set of Microsoft's \acrshort{API}s available within the Windows Operating System. It provides basic \acrshort{UI}, Environment handling, data access and storage utilities and more.~\cite{windowsAPI}}}
\newglossaryentry{Fragment}{name={Fragment},description={A 1-bit configurable component of an FPGA. Fragments do not exist however they are used as a conceptual aid to improve understanding of configuration process.}}
\newglossaryentry{gateArray}{name={Gate Array},description={The primary structure of an \gls{FPGA} device. Composed of a regular arrangement of logic gates.}}
\newglossaryentry{positionMedian}{name={Position Median},description={The computed center of an \acrshort{FPGA} design}}
\newglossaryentry{scatterScore}{name={Scatter Score},description={A numeric representation of the clustering of a design}}
\makeglossaries


\begin{document}

% Front Matter
\input frontmatter/fm

\newpage

	\startfirstchapter{Introduction}
\label{chapter:introduction}
The term \textit{Trojan Horse} or \textit{Trojan} has become a modern metaphor for a deception where by an unsuspecting victim welcomes a foe into an otherwise safe environment.~\cite{searchForTrojanWar}
%Grand tales have been told of large wooden horses though scholars believe the true trojan was far more practical.\cite{searchForTrojanWar}
%Covered battering rams were commonly used at the time for gaining access to fortified areas. 
%Such a device is thought to be the real Trojan horse.
Though modern civilization rarely has need for large walls we are similarly surrounded.
Not by stone and mortar but by the technology we so heavily rely on.
These days it is more common to come across a piece of equipment with some form of computer in it than without.
They provide us entertainment, education, security, monitor our health, grow our food and more.
Our reliance make us susceptible to their compromise.
Since the dawn of the computer we have dealt with software threats; we are almost as good as protecting ourselves against them as they are at attacking us.
In recent years a new incarnation of danger has emerged; in hardware.
In this new arena of attack and defend those who seek to defend are far behind.

\section{Motivation}
In the summer of 2007 an Israeli military action referred to as Operation Orchard commenced.
A group of F-15I Ra'am fighter jets from the Israeli Air Force 69th Squadron took off to attack a suspected nuclear reactor in neighboring Syria.
In the flight path was a Syrian radar station which boasted 'state-of-the-art' aircraft detection and neutralization technology. 
The Israeli war planes were able to approach and destroy the installation undetected.
Though never proven it is commonly accepted that the detection mechanism was deactivated by a back-door circuit inserted into the radar system.~\cite{stoppingHTsIEEESpectrum}
In 2011 over 1300 cases were reported to the \acrfull{ERAI} of modified \acrshort{IC}s and the occurrence has been going up.~\cite{counterfeitIEEESpectrum}
\acrfull{ICs} are a large part of modern life yet it is easy to forget that they drive virtually every piece of technology used today.
Ensuring their \acrshort{IC}s run our devices as expected is vital in the digital era.
Since their discovery there has been concerted effort to detect hardware trojans but the innate complexity of \acrshort{IC}s makes it difficult.~\cite{hardwareTrojanSurvey2015}

\section{Contributions}
\section{Organization}


%The main goal of this document was to set up competently the Latex framework for a thesis document at UVic, with all the assorted details of front pages, numbering and so on. In order to present such a template to other users, the content has been filled from the guidelines of an experienced graduate advisor and supervisor on how to write a thesis in the first place.
%
%The Latex framework given in these files should work in the Windows, Mac and Unix environments. In Windows, the development environment is MikTex, available as a free download from the web page at \textit{miktex.org}. Miktex includes all the tools for \LaTeX, without a convenient editor. I have been using
%\textit{WinEdt} which is still the best as far as working in a well integrated manner with MikTex, even if one has to pay a small price. \textit{WinEdt} is available from \textit{winedt.com}. The corresponding package for the Mac is MacTex, available from \textit{www.tug.org$\backslash$mactex}.
%Please note that in the actual file with the \LaTeX code there are extra local comments referring to small differences between platforms. This is especially crucial for the insertion of figures of any type (see also Chapter 3).
%
%The package \textit{BibTex} is used for the automatic bibliography and it is incorporated in the MikTex package. The style of bibliography exemplified is this document is the "plain" one,
%most often used in science theses. This is shown
%by the entry \textit{plain} in the \textit{bibliographystyle}
%command which can be found in the top file, namely
%\textit{mainthesisUVIC.tex}. Substitute the
%appropriate bibliography style for your needs by finding
%the correct parameter. In order to help you there is a PDF
%file entitled: "InformationOnBibliographyStyles.pdf" in the
%same folder as the top file \textit{mainthesisUVIC.tex}.
%
%The title pages are correctly formatted according to the guidelines. However the superset is given here as a template for a Ph.D. dissertation and adjustments must be made for a Master's thesis accordingly.
%
%When presented with the task of having to write a large document reporting on the work done for the thesis portion of a graduate degree most people have anxiety attacks. It seems that the research work has been done, probably with great competence, yet one feels totally lost on how to approach the writing part - unless of course one is a writer to start with. While I am not at all the expert, I have experience both in writing documents and in supervising many graduate students. I have developed a simple enough structure which guides me and my students most of the times fairly well. We then customize it for different audiences and situations. The content of this pseudo mini thesis summarizes the structure I use.
%
%This document, however, is \textit{not at all} a Latex manual. There are a few examples sprinkled here and there of various commands. However no explanations or full examples are given for how to use Latex itself. That is most certainly left to the user!
%
%Starting at the very beginning, my first step is to set up the formatting framework before the content itself, just like the frame of a building is the sustaining structure. By having the content only in bullet points which can be moved and changed at will I can have a whole document which will always compile and be ready for presentation. I ask my student to prepare mock chapters which contains only these bulleted lists to be used for discussion. The bulleted list I had for this chapter is as follows:
%\begin{enumerate}
%        \item State the dual purpose of this document (just mention briefly);
%        \item State what this document is not (just mention);
%        \item Describe what I think should go in the Introduction in general;
%        \item Describe what I think should not go in the "Introduction" chapter in general;
%        \item Describe the structure of the document.
%\end{enumerate}
%
%When the writing for a chapter is particularly long, it might be better to have some, if not all, sections in separate files. It makes it a lot easier to find possible errors in Latex as well, since the command to input a file for a separate section can always be commented out. When looking at the \textit{.tex} file itself for this introduction chapter you will notice that there are 2 sections, namely "How to Start an Introduction" and "Is a Review of All previous Work Necessary Here?" which are included as separate files, immediately following this paragraph.
%\section{How to Start an Introduction}
%
%
%What is the difference between an "Introduction" and the "Background" for a thesis? Should they be together? What about the review of the work from other people?
%
%These are important questions which must be discussed between a student and the supervisor.
%
%My view is that the introduction should be exactly that: a short introduction, not the history of the problem, its content and all its solutions to date. One major ingredient \textit{must} be a marketing angle, such that the reader becomes deeply motivated to continue on with the rest of the document. When I referee an article I usually read the abstract, the introduction and the conclusion. At that point I expect to be able to state to someone else what the work is about, what seems to be the new advancement and how interesting I think it will be to read the rest.
%
%My main statement about the introduction is:
%\textit{"keep it short and write it last".} The main features should be as follows:
%\begin{enumerate}
%	\item {3-4 pages at most;}
%	\item {Start with a VERY short statement of the problem (2-3 sentences) - the problem should be stated, not described, as there will be a whole chapter for that;}
%	\item {State why the problem is important, its impact, how well it has been studied recently, its application (3 sentences) - this should be again a brief motivation, leaving a full impact description to later in the document;}
%	\item {Give a sketch of the new approach - there will be a whole chapter with all the details, now just impress the reader about what is the new approach, just as you would do if your boss asked you at work during an elevator ride;}
%	\item {Sketch the main new ideas of the new approach - again briefly, just get the reader interested;}
%	\item {Give a short statement regarding the results, nothing too elaborate, but certainly you should blow your horn and make sure that the reader is intrigued;}
%	\item {Interspersed in all the writing above do not forget the marketing angle, trying to suggest forcefully why the reader should keep reading;}
%	\item {Give an outline of what is to come in the organization of the thesis overall - you will find one below for this document.}
%	
%\end{enumerate}
%
%Finally the strong suggestion is write the introduction chapter last. It will be faster, you will know what to say as the rest is already there, and the abstract, introduction and conclusion will be a mirror and complement of each other. You may well ask where to start writing your thesis. My view is included in the organization below.
%\section{Is a Review of All Previous Work Necessary Here?}
%
%Many people like to place in the introductory chapter a review of the work from everybody else on the same problem. I find this utterly boring and counter-productive. If I am an expert in the field I probably know all this and could even write it better, so the last thing I want to read at the beginning is a history of research. If I am not an expert in the field, I would prefer to read up all the details of the problem itself and understand its context before I can even take in any ideas of what others have done.
%
%Secondly, often when talking about the work of others one includes a bit judgment on it. This may be necessary, as the pivot of the new work may indeed be that there was an open problem left unsolved by the other researchers. Yet it is hard not to sound negative, to describe the work of others stating that their solution did not include some important part without denigrating. If one of those researchers is a reader and perhaps an examiner, they do not want to start reading by being told that their own work is inadequate.
%
%I suggest a general summary of no more than 3-4 paragraphs of the work in the area with references, just to give the impact of the importance of the problems to be solved. Use then a "just-in-time" approach, where the relevant work of another researcher is described when needed in a particular step of the exposition of the problem or of the new solution. I normally place a much more extensive review of the work of others after I have presented my solution and the results of my experimental or formal proof work, so that I can analyze and compare effectively without boring repetitions and without writing a literature review on a subject. Many good articles in journals follow this scheme and most grant organization ask for a literature review to be included at the end of an application. First they want to be impressed with the proposal for new work!
%
%A similar logic should be applied to background knowledge and definitions. This is especially true in a scientific thesis where often one find a whole set of mathematical definitions lumped together in chapter 1, yet not used until chapter 6, by which time the reader has completely forgotten it and needs to shuffle back with irritation. Use again a "just-in-time" approach and give definitions and explanations locally, the first time they are needed.
%
%%\pagebreak
%\section{My Claims}
%Something must be new in this work, no matter how small, since you are getting a graduate degree for it! Tell me about it clearly and succintly right now, just as you did in the abstract. Make an impact here. How about something like the following box:
%
%I make \textit{four} claims which
%my dissertation validates:
%\\
%
%\framebox{%
%\parbox{5in}{
%	My new algorithm to solve the problem of doing nothing include these important new features whose practical applicability can be proved both formally and empirically:
%	\begin{enumerate}
%	\item first feature;
%	\item second feature;
%	\item everything is much easier to understand, and therefore, easier to implement correctly.
%	\end{enumerate}
%}}
%\\
%
%\noindent Claim 1 and claim 2 are \textit{quantitative} - they will be proven by experiment.
%
%\noindent Claim 3 is \textit{qualitative} - they will be demonstrated by argument.
%
%\subsection{The Importance of My Claims}
%
%Some very important positive consequences
%arise from the validation of the above claims.
%It is these consequences that comprise a significant
%positive contribution to research in the field
%of whatever the field is.
%\\
%
%\noindent Claim 1 implies that:
%\begin{enumerate}
%\item{Something profound which applies to:
%	\begin{itemize}
%	\item {something excellent;}
%	\item {something important.}
%	\end{itemize}}
%\item{Something else just as profound.}
%\end{enumerate}
%
%\noindent Claim 2 implies that:
%\begin{itemize}
%\item{Repeat as above if necessary and useful.}
%\end{itemize}
%
%\noindent The consequence of claim 3 is that:
%\begin{itemize}
%\item{There must be something good coming out of all this work!}
%\end{itemize}
%
%\section{Agenda}
%
%This section provides a map of the dissertation
%to show the reader where and how it validates
%the claims previously made. Here is where I am also presenting my own style of organization which may be totally different from what your supervisor thinks. However, trust me, this is a good solid beginning for a structure. Your supervisor may ask you to change it, but will still appreciate what you have! For each of the chapters below I also give a short summary of what the main focus should be and then I expand on it  a bit within the chapter itself.
%
%\begin{description}
%\item[\textbf{Chapter 1}] contains a statement of
%the claims which will be proved by this dissertation followed by an overview of the structure of the document itself.
%\item[\textbf{Chapter 2}] describes in details the open problem which is to be tackled together with its context, its impact and the overall motivation for the research overall.
%\item[\textbf{Chapter 3}] gives the new research, its methodology, the algorithms involved, the new solution, the new work done. Formal proofs and arguments are made here. This is the first of the two contributions expected in a thesis for a graduate degree.
%\item[\textbf{Chapter 4}] is where the experiments and the methodology for them is fully described. The first part includes all details of how the empirical side of the research has been conducted. Note that not every thesis has this empirical portion.
%\item[\textbf{Chapter 5}] includes the evaluation of the data presented above and the comparisons with the work of others, to show how much better the new approach is. This is the second of the two contributions expected in a thesis for a graduate degree. Note that this part could be consolidated into the chapter above.
%\item[\textbf{Chapter 6}] contains a restatement of the claims and results of the dissertation. It also enumerates avenues of future work for further development of the concept and its applications.
%\end{description}
%
%The list above is not complete. Chapter 3 actually includes a lot more, as I could not resist placing in it a few \LaTeX examples to help you along. This document is not a primer for \LaTeX, but there is no harm done in giving a little help.

	\startchapter{Hardware Trojans}
\label{chapter:hardwareTrojans}

\newlength{\savedunitlength}
\setlength{\unitlength}{2em}
\section{Background} \label{sec:trojanBackGround}
\acrfull{ICs} are continuously decreasing in size whilst increasing in complexity.
These trends require ever more people and sophisticated means of manufacture which in turn creates security vulnerabilities.
Products developed by semiconductor companies tend to be compromised in one of two ways.
First, due to its complexity it is rare for an \acrshort{IC} to be entirely manufactured within a single company.
Frequently, steps in the production-chain are outsourced.
It is within these 'third-party' contributors that products can be maliciously modified.
Secondly, for various reasons, employees of trusted contributors have been known to make modifications~\cite{trojanSurvey2014}.
\acrshort{IC}s are an integral part of every facet of the modern world.
Proper application of a trojan can provide information, control of mechanical systems, surveillance and more to an unauthorized party.


\section{Topology} \label{sec:topology}
The discussion, detection and evaluation of hardware trojans requires a comprehensive means of description.
Several hardware trojan taxonomies have been proposed~\cite{taxonomy1, taxonomy2, taxonomy3, taxonomy4}.
In~\cite{taxonomy1}, trojans were organized based solely on their activation mechanisms.
A taxonomy based on the location, activation and action of a trojan was presented in \cite{taxonomy2}, \cite{taxonomy3}.
However, these approaches do not consider the manufacturing process.
Another taxonomy was proposed in~\cite{taxonomy4} which employs five categories: insertion, abstraction, activation, effect, and location.
While this is more extensive than previous approaches, it fails to account for the physical characteristics of a trojan.
An additional taxonomy was proposed in~\cite{samerAttribute} which considers all attributes a hardware trojan may posses.
This taxonomy is the most comprehensive and was selected as the means of description for this work.
\begin{figure}
	\centering
	\includegraphics[width=1.0\linewidth]{figures/HW_trojan}
	\caption{The thirty-three attributes of the hardware trojan taxonomy in~\cite{samerAttribute}.}
	\label{fig:HW_trojan}
\end{figure}
It is comprised of thirty-three attributes organized into eight categories as shown in Fig.~\ref{fig:HW_trojan}.
These categories can be arranged into the following four levels as indicated in Fig.~\ref{fig:trojan_life_cycle}.
\begin{enumerate}
	\item The \textbf{insertion} (chip life-cycle) level/category comprises the attributes pertaining to the IC production stages.
	\item The \textbf{abstraction} level/category corresponds to where in the IC abstraction the trojan is introduced.
	\item The \textbf{properties} level comprises the behavior and physical characteristics of the trojan.
	\item The \textbf{location} level/category corresponds to the location of the trojan in the IC.
\end{enumerate}
The properties level consists of the following categories.
\begin{itemize}
	\item The \textbf{effect} describes the disruption or effect a trojan has on the system.
	\item The \textbf{logic type} is the circuit logic that triggers the trojan, either combinational or sequential.
	\item The \textbf{functionality} differentiates between trojans which are functional or parametric.
	\item The \textbf{activation} differentiates between trojans which are always on or triggered.
	\item The \textbf{layout} is based on the physical characteristics of the trojan.
\end{itemize}
\begin{figure}
	\centering
	\includegraphics[width=0.8\linewidth]{figures/trojan_life_cycle}
	\caption{The hardware trojan levels \cite{samerAttribute}.}
	\label{fig:trojan_life_cycle}
\end{figure}

\begin{table*}
	\[
	\newcommand\scalemath[2]{\scalebox{#1}{\mbox{\ensuremath{\displaystyle #2}}}}
	\mathbf{R} =
	\left[
	\scalemath{0.5}{
		\begin{array}{l||c|c|c|c|c||c|c|c|c|c|c||c|c|c|c|c|c|c|c|c|c|c|c|c|c|c|c|c||c|c|c|c|c}
		A & 1$~$ & 2$~$  & 3$~$ & 4$~$& 5$~$ & 6$~$  & 7$~$ & 8$~$ & 9$~$ & 10  & 11 & 12& 13 & 14  & 15 & 16& 17 & 18  & 19 & 20& 21 & 22  & 23 & 24& 25 & 26  & 27 & 28 & 29 &30 & 31 & 32 & 33\\ \hline \hline
		1 & 0 & 1 & 0 & 0 & 0 & 1 & 0 & 0 & 0 & 0 & 0 &  &  &  &  &  &  &  &  &  &  &  &  &  &  &  &  &   &  &  &  &  &  \\ \hline
		2 & 0 & 0 & 1 & 0 & 0 & 0 & 1 & 0 & 0 & 0 & 0 &  &  &  &  &  &  &  &  &  &  &  &  &  &  &  &  & &  &  &  &  &   \\ \hline
		3 & 0 & 0 & 0 & 1 & 0 & 0 & 0 & 0 & 0 & 0 & 1 &  &  &  &  &  &  &  &  &  &  &  &  &  &  &  &  & &  &  &  &  &   \\  \hline
		4 & 0 & 0 & 0 & 0 & 1 & 1 & 0 & 0 & 1 & 0 & 0 &  &  &  &  &  &  &  &  &  &  &  &  &  &  &  &  &&  &  &  &  &    \\  \hline
		5 & 0 & 0 & 0 & 0 & 0 & 1 & 0 & 0 & 0 & 0 & 0 &  &  &  &  &  &  &  &  &  &  &  &  &  &  &  &  &  &  &  &  &  &  \\  \hline \hline
		6 &  &  &  &  &  & 0 & 1 & 0 & 0 & 0 & 0 &1 & 1 & 0 & 1 & 0 & 0 & 1 & 1 & 1 & 0 & 0 & 0 & 0 & 0 & 0 & 0 & 0  &  &  &  &  &  \\ \hline
		7 &  &  &  &  &  & 0 & 0 & 1 & 0 & 0 & 0 &1 & 0 & 0 & 1 & 1 & 1 & 1 & 0 & 1 & 1 & 1 & 1 & 1 & 0 & 0 & 0 & 0  &  &  &  &  &   \\ \hline
		8 &  &  &  &  &  & 0 & 0 & 0 & 1 & 0 & 0 &1 & 0 & 0 & 1 & 1 & 1 & 1 & 0 & 1 & 1 & 1 & 1 & 1 & 1 & 1 & 1 & 1  &  &  &  &  &   \\ \hline
		9 &  &  &  &  &  & 0 & 0 & 0 & 0 & 1 & 0 & 1 & 0 & 0 & 1 & 1 & 1 & 1 & 0 & 1 & 1 & 1 & 0 & 0 & 0 & 0 & 0 & 0 &  &  &  &  &    \\ \hline
		10 &  &  &  &  &  & 0 & 0 & 0 & 0 & 0 & 1 & 1 & 0 & 1 & 0 & 0 & 1 & 1 & 1 & 1 & 0 & 0 & 0 & 1 & 0 & 1 & 1 & 0  &  &  &  &  &  \\  \hline
		11 &  &  &  &  &  & 0 & 0 & 0 & 0 & 0 & 0 &1 & 1 & 1 & 0 & 0 & 0 & 1 & 1 & 1 & 0 & 0 & 1 & 1 & 1 & 1 & 1 & 1  &  &  &  &  &    \\  \hline \hline
		12 & &  &  &  &  &  &  &  &  &  &  & 0 & 0 & 0 & 0 & 1 & 1 & 1 & 0 & 1 & 1 & 1 & 1 & 1 & 1 & 1 & 1 & 1&1 & 1 & 1 & 1 & 1 \\ \hline
		13 & &  &  &  &  &  &  &  &  &  &  &0 & 0 & 0 & 0 & 0 & 1 & 0 & 1 & 1 & 0 & 1 & 0 & 1 & 0 & 1 & 1 & 1&1 & 1 & 1 & 1 & 1 \\ \hline
		14 &&  &  &  &  &  &  &  &  &  &  & 0 & 0 & 0 & 0 & 0 & 0 & 0 & 1 & 1 & 0 & 0 & 0 & 1 & 0 & 1 & 1 & 1&1 & 1 & 1 & 1 & 1 \\ \hline
		15 & &  &  &  &  &  &  &  &  &  &  &0 & 0 & 0 & 0 & 1 & 1 & 1 & 0 & 1 & 1 & 1 & 1 & 1 & 1 & 1 & 1 & 1&1 & 0 & 1 & 1 & 1 \\ \hline
		16 & &  &  &  &  &  &  &  &  &  &  &1 & 0 & 0 & 1 & 0 & 0 & 1 & 0 & 0 & 1 & 1 & 1 & 0 & 1 & 1 & 1 & 1&1 & 1 & 1 & 1 & 1 \\ \hline
		17 & &  &  &  &  &  &  &  &  &  &  &1 & 1 & 0 & 1 & 0 &0 & 1 & 0 & 1 & 1 & 1 & 1 & 1 & 1 & 1 & 1 & 1&1 & 1 & 1 & 1 & 1 \\ \hline
		18 & &  &  &  &  &  &  &  &  &  &  &1 & 0 & 0 & 1 & 1 & 1 & 0 & 0 & 1 & 1 & 1 & 1 & 1 & 1 & 1 & 1 & 1&1 & 1 & 1 & 1 & 1 \\ \hline
		19 & &  &  &  &  &  &  &  &  &  &  &0 & 1 & 1 & 0 & 0 & 0 & 0 & 0 & 1 & 0 & 0 & 0 & 1 & 0 & 1 & 0 & 1&0 & 0 & 1 & 1 & 1 \\ \hline
		20 & &  &  &  &  &  &  &  &  &  &  &1 & 1 & 1 & 1 & 0 & 1 & 1 & 1 & 0 & 0 & 0 & 1 & 1 & 1 & 1 & 1 & 1&1 & 1 & 1 & 1 & 1 \\ \hline
		21 & &  &  &  &  &  &  &  &  &  &  &1 & 0 & 0 & 1 & 1 & 1 & 1 & 0 & 0 & 0 & 0 & 1 & 1 & 0 & 1 & 1 & 1&1 & 1 & 1 & 1 & 1 \\ \hline
		22 & &  &  &  &  &  &  &  &  &  &  &1 & 1 & 0 & 1 & 1 & 1 & 1 & 0 & 0 & 0 & 0 & 0 & 1 & 0 & 1 & 1 & 0&1 & 1 & 1 & 1 & 1 \\ \hline
		23 & &  &  &  &  &  &  &  &  &  &  &1 & 0 & 0 & 1 & 1 & 1 & 1 & 0 & 1 & 1 & 0 & 0 & 0 & 1 & 1 & 1 & 1&1 & 1 & 0 & 0 & 0  \\ \hline
		24 & &  &  &  &  &  &  &  &  &  &  &1 & 1 & 1 & 1 & 0 & 1 & 1 & 1 & 1 & 1 & 1 & 0 & 0 & 0 & 1 & 1 & 0&1 & 1 & 1 & 1 & 1 \\ \hline
		25 &&  &  &  &  &  &  &  &  &  &  & 1 & 0 & 0 & 1 & 1 & 1 & 1 & 0 & 1 & 0 & 0 & 1 & 0 & 0 & 1 & 1 & 0& 1 & 1 & 1 & 1 & 1\\ \hline
		26 & &  &  &  &  &  &  &  &  &  &  &1 & 1 & 1 & 1 & 1 & 1 & 1 & 1 & 1 & 1 & 1 & 1 & 1 & 1 & 0 & 1 & 1& 1 & 1 & 1 & 1 & 1\\ \hline
		27 & &  &  &  &  &  &  &  &  &  &  &1 & 1 & 1 & 1 & 1 & 1 & 1 & 0 & 1 & 1 & 1 & 1 & 1 & 1 & 1 & 0 & 0 &1 & 1 & 1 & 1 & 1\\ \hline
		28 & &  &  &  &  &  &  &  &  &  &  &1 & 1 & 1 & 1 & 1 & 1 & 1 & 1 & 1 & 1 & 0 & 1 & 0 & 0 & 1 & 0 &0 &1 & 1 & 1 & 1 & 1\\ \hline \hline
		29 & &  &  &  &  &  &  &  &  &  &  & &  &  & &  &  &  &  &  &  &  &  &  &  &  &  & & &  &  &  & \\ \hline
		30 & &  &  &  &  &  &  &  &  &  &  & &  &  & &  &  &  &  &  &  &  &  &  &  &  &  & & &  &  &  & \\ \hline
		31 & &  &  &  &  &  &  &  &  &  &  & &  &  & &  &  &  &  &  &  &  &  &  &  &  &  & & &  &  &  & \\ \hline
		32 & &  &  &  &  &  &  &  &  &  &  & &  &  & &  &  &  &  &  &  &  &  &  &  &  &  & & &  &  &  & \\ \hline
		33 & &  &  &  &  &  &  &  &  &  &  & &  &  & &  &  &  &  &  &  &  &  &  &  &  &  & & &  &  &  & \\
		\end{array}
	}
	\right ]
	\label{ss1}
	\]
\end{table*}
The relationships between the trojan attributes shown in Fig.~\ref{fig:HW_trojan} can be described using a matrix $\mathbf{R}$~\cite{samerAttribute}.
%Systematic analysis of matrix $\mathbf{R}$ provides useful insight into the overall nature of a subject.
%A procedure used to develop the nature of a subject has been described in~\cite{samerAttribute} and is referred to as Classification.
Entry $r(i,j)$ in $\mathbf{R}$ indicates whether or not attribute $i$ can lead to attribute $j$.
For example, $r(2,3) = 1$ indicates that design (attribute 2) can lead to fabrication (attribute 3).
This implies that if an IC can be compromised during the design phase (attribute 2), it may influence the fabrication phase (attribute 3).

The matrix $\mathbf{R}$ is divided into sub matrices as follows
\[
\newcommand\scalemath[2]{\scalebox{#1}{\mbox{\ensuremath{\displaystyle #2}}}}
\mathbf{R} =\left[
\scalemath{1.1}{
	\begin{array}{l*{3}{c}}
	\mathbf{R_1} ~ & ~ \mathbf{R_{12}} ~ & ~ 0 ~  &  ~ 0   \\
	0         & \mathbf{R_2}      &\mathbf{R_{23}}       & ~ 0 \\
	0          & 0           & \mathbf{R_3}          & ~ \mathbf{R_{34}} \\
	0          & 0           & 0                & ~ \mathbf{R_4} \\
	\end{array}
}
\right]
\label{R}
\]
where $\mathbf{R_1}$, $\mathbf{R_2}$, $\mathbf{R_3}$ and $\mathbf{R_4}$ indicate the attribute relationships within a category.
For example, $\mathbf{R_1}$ is given by
\[
\newcommand\scalemath[2]{\scalebox{#1}{\mbox{\ensuremath{\displaystyle #2}}}}
\mathbf{R_1} =\left[
\scalemath{1.0}{
	\begin{array}{l|*{11}{c}}
	A & 1$~$ & 2$~$  & 3$~$ & 4$~$& 5$~$\\ \hline
	1 & 0 & 1 & 0 & 0 & 0  \\
	2 & 0 & 0 & 1 & 0 & 0  \\
	3 & 0 & 0 & 0 & 1 & 0  \\
	4 & 0 & 0 & 0 & 0 & 1  \\
	5 & 0 & 0 & 0 & 0 & 0  \\
	\end{array}
}
\right ]
\label{R1}
\]
Submatrix $\mathbf{R_{12}}$ relates the attributes of the insertion category to the attributes of the abstraction category.
An example of this submatrix is
\[
\newcommand\scalemath[2]{\scalebox{#1}{\mbox{\ensuremath{\displaystyle #2}}}}
\mathbf{R_{12}} =\left[
\scalemath{1.0}{
	\begin{array}{l|*{11}{c}}
	A & 6$~$  & 7$~$ & 8$~$ & 9$~$ & 10  & 11\\ \hline
	1  & 1 & 0 & 0 & 0 & 0 & 0 \\
	2  & 0 & 1 & 0 & 0 & 0 & 0 \\
	3  & 0 & 0 & 0 & 0 & 0 & 1 \\
	4  & 1 & 0 & 0 & 1 & 0 & 0 \\
	5  & 1 & 0 & 0 & 0 & 0 & 0 \\
	\end{array}
}
\right ]
\label{R12}
\]

%%%%%%%%%%%%%%%%%%%%%%%%%%%%%%%%%%%%%%%%%%%%%%%%%%
\section{Hardware Trojan Analysis Techniques} \label{sec:techniques}
%\color{red}
The benefit of the taxonomy described in section~\ref{sec:topology} is that its comprehensive and structured design provides the ability to develop systematic methods of analysis.
A pair of techniques have already been proposed that provide additional insight.

\subsection{Classification}  \label{sec:techniqueClassification}
Attributes of a trojan are not always immediately observable.
The complex nature of \acrshort{IC}s leaves a lot of room for difficulty extracting them all.
Those that can be extracted are used as input to a systematic process of scanning the rows and columns of matrix $\mathbf{R}$ to infer the existence of others~\cite{samerAttribute,samerDissertation}.
It can be used to determine the possible locations of a trojan within the design and in which manufacturing phases it can be inserted.
Conversely, the phase a trojan was inserted can be used to determine which abstraction levels are vulnerable,
the trojan properties, and what locations can be compromised.
To easily understand the characteristics of a trojan, a directed graph is generated from $\mathbf{R}$.
Attributes are represented by nodes and their relationships by edges.\newline\newline
Consider a trojan that has the following attributes:
\begin{itemize}
	\item causes denial of service (DoD) (attribute 15),
	\item composed of combinational logic (attribute 17),
	\item functional (attribute 18), and
	\item externally triggered (attribute 22).
\end{itemize}
An examination of $\mathbf{R}$ leads to the graph shown in Fig.~\ref{fig:full2}.
If it is determined that it is not possible to insert the trojan in the design phase (attribute 2), then
attribute 2 can be removed from the graph.
Without a direct connection to attribute 1, attributes 3, 4 and 5 must also be removed.
Without attribute 2 the trojan can only be inserted in the specification phase (attribute 1).
These relationships are cataloged in matrix $\mathbf{R}$ as described in section~\ref{sec:topology} but the directed graph provides a much more human-friendly visual.
\begin{figure}[]
	\centering
	\includegraphics[width=0.6\linewidth]{figures/full2}
	\caption{The directed graph corresponding to a trojan.}
	\label{fig:full2}
\end{figure}
%%%%%%%%%%%%%%%%%%%%%%%%%%%%%
\subsection{Trojan Evaluation} \label{sec:techniqueEvaluation}
Due to the complexity of IC designs, hardware trojans are typically unique.
As a consequence, detection methods developed thus far have been developed to detect specific trojans.
The diversity in both trojans and detection methods makes it difficult to evaluate, compare, and organize them.
A means of evaluating hardware trojans and detection methods based on the eight attribute categories was proposed in~\cite{samerClassDetection}.
A trojan or detection method will possess some combination of attributes from each of the eight categories, and
each combination is assigned two numerical values.
The value $I$ is used to identify the combination, while the
value $C$ is used to denote the severity (for a trojan) or coverage (for a detection method) of the combination.
Tables of $I$ and $C$ values for the eight categories were presented in~\cite{samerDissertation}.
For example, the logic type category describes the circuit logic which activates the trojan.
Table~\ref{tbl:logicTypeTable} shows the possible attribute combinations for this category and the corresponding values of $I_L$ and $C_L$.
\begin{table}[h]
	\centering
	\caption{Logic Type Category Values}
	\label{tbl:logicTypeTable}
	\begin{tabular}{|c|c|c|}
		\hline
		Attributes & $I_L$ & $C_L$ \\ \hline
		Sequential (16) & 2 & 2 \\
		Combinational (17) & 1 & 1 \\
		Both (16 and 17)& 3 & 3 \\ \hline
	\end{tabular}
\end{table}

The $I$ and $C$ values from the category tables are arranged into identification and severity/coverage vectors, respectively.
For a trojan the vectors are denoted $I_T$ and $C_T$, and for a detection method they are denoted $I_D$ and $C_D$.
Thus, an identification vector is
\[
I = I_RI_AI_EI_LI_FI_CI_PI_O,
\]
where {$I_R,I_A,I_E,I_L,I_F,I_C,I_P,I_O$} are the
\{Insertion, Abstraction, Effect, Logic Type, Functionality, Activation, Physical Layout, Location\} category identification values, respectively, and
a severity/coverage vector is
\[
C = C_RC_AC_EC_LC_FC_CC_PC_O,
\]
where {$C_RC_AC_EC_LC_FC_CC_PC_O$} are the
\{Insertion, Abstraction, Effect, Logic Type, Functionality, Activation, Physical Layout, Location\} category strength values, respectively.

\begin{table}[h]
	\centering
	\caption{Identification and Severity Vectors for Two Hardware Trojans}
	\label{tbl:severityTable}
	\renewcommand{\arraystretch}{1.2}
	\begin{tabular}{|c|p{3mm}p{3mm}p{3mm}p{3mm}p{3mm}p{3mm}p{3mm}p{3.5mm}|p{3mm}p{3mm}p{3mm}p{3mm}p{3mm}p{3mm}p{3mm}p{3.5mm}|}
		\hline
		Technique & \multicolumn{8}{c|}{Parameters ($I_P$)} & \multicolumn{8}{c|}{Severity ($C_P$)} \\ \cline{2-17}
		& $I_R$ & $I_A$ & $I_E$ & $I_L$ & $I_F$ & $I_C$ & $I_P$ & $I_O$ & $C_R$ & $C_A$ & $C_E$ & $C_L$ & $C_F$ & $C_C$ & $C_P$ & $C_O$ \\ \hline
		Trojan A~\cite{samerAttribute} & 2 & 6 & 2 & 1 & 2 & 1 & 7 & 7 & \textbf{\textcolor{red}{2}} & 6 & 4 & 1 & 2 & 1 & 5 & 2 \\ \hline
		Trojan B~\cite{samerAttribute} & 3 & 3 & 1 & 2 & 1 & 2 & 8 & 1 & \textbf{\textcolor{red}{3}} & 3 & 2 & 2 & 1 & 3 & 6 & 1 \\ \hline
	\end{tabular}
\end{table}
\begin{table}[h]
	\centering
	\caption{Identification and Coverage Vectors for Two Hardware Trojan Detection Methods}
	\label{tbl:detectionTable}
	\renewcommand{\arraystretch}{1.2}
	\begin{tabular}{|c|p{3mm}p{3mm}p{3mm}p{3mm}p{3mm}p{3mm}p{3mm}p{3.5mm}|p{3mm}p{3mm}p{3mm}p{3mm}p{3mm}p{3mm}p{3mm}p{3.5mm}|}
		\hline
		Technique & \multicolumn{8}{c|}{Parameters ($I_P$)} & \multicolumn{8}{c|}{Coverage ($C_P$)} \\ \cline{2-17}
		& $I_R$ & $I_A$ & $I_E$ & $I_L$ & $I_F$ & $I_C$ & $I_P$ & $I_O$ & $C_R$ & $C_A$ & $C_E$ & $C_L$ & $C_F$ & $C_C$ & $C_P$ & $C_O$ \\ \hline
		\cite{method1} & 3 & 3 & B & 1 & 2 & 4 & 7 & V  & 3 & 3 & \textbf{\textcolor{red}{7}} & 1 & 2 & 3 & 5 & 5 \\ \hline
		\cite{method2} & 3 & 3 & 1 & 2 & 1 & 4 & 7 & V  & 3 & 3 & \textbf{\textcolor{red}{2}} & 3 & 1 & 3 & 5 & 5 \\ \hline
	\end{tabular}
\end{table}

Table~\ref{tbl:severityTable} provides a comparison of two hardware trojans.
Trojan A has a lower severity than Trojan B in the insertion category, denoted by $C_R$.
This indicates that Trojan B can be inserted in more stages of the manufacturing process than Trojan A.
Table~\ref{tbl:detectionTable} gives a comparison between two detection methods.
The method in \cite{method1} has a higher coverage in the effect category ($C_E$) than the method in \cite{method2},
indicating that it can detect more trojans based on their effects.
\section{Recent Work}
Hardware trojans are a new and exciting field of study.
As \acrshort{FPGA}s take a larger portion of the \acrfull{IC} market the need for security becomes greater.
With this, detection and analysis of trojans in \acrshort{FPGA}s has become a growing topic and has seen some development in recent years.
The majority of work focused on \acrshort{FPGA} trojans has employed either a means of reverse engineering, functional testing or 'side-channel' analysis.
The configuration \gls{Bitstream} contains all of the information one would want to know about what is happening on the device.
There has been little effort to directly analyze the \gls{Bitstream} because manufacturers are reluctant to provide intimate details of its format.
  
\subsection{\acrfull{CRC} Detection} \label{sec:crcDetection}
Because \acrshort{FPGA} design is done using a software language it is possible for designers to share component designs easily.
It is common for designers to employ 'third-party' \acrfull{IP}. 
This sharing of component design provides the opportunity for attackers to add trojans to commercial products by sharing trojan-containing \acrshort{IP}.
In 2013 researchers at Cairo University proposed a method of insulating externally sourced \acrshort{IP} with \acrfull{CRC} defense modules~\cite{crcDetection}.
According to the authors their method is capable of detecting leaked information with a 99.95\% accuracy.
This method employs a methodology referred to as 'built-in-self-test" (BIST) where additional hardware is added to the design.
The additional hardware performs run-time checking of circuit output.
In general, the BIST method is only capable of detecting modifications the designers have foreseen.
In this case, this method is only capable of detecting trojans that possess the attribute Information Leakage (13) from the taxonomy presented ins section~\ref{sec:topology}.
In addition the authors report considerable detriment to power consumption and performance.


\subsection{\acrfull{RO} Detection}
Researchers at the Technological Educational Institute of Western Greece and Industrial Systems Institute/RC “Athena” jointly proposed a method of using \acrfull{ROs} as a mechanism for detecting hardware trojans~\cite{ringOscillatorMethod}.
A \acrshort{RO} is a circuit composed of inverters formed into a loop.
Electric current looping through the \acrshort{RO} does so at an inherent frequency.
By configuring the circuit paths of the user's design into a \acrshort{RO} it is possible to create a 'signature'.
This signature is an expected frequency emitted from the desired design. 
The authors claim that modifications to the design will alter the frequency emitted by its circular configuration.
The experimental results showed that modifications did in-fact alter the frequency enough to reliably detect modifications.
This method can reliably detect hardware trojans but is incapable of providing any details regarding its effect.
Further, the stipulation that the desired design must be such that it forms an oscillating ring is impractical.
It is impossible to guarantee that all real-world designs can form an \acrshort{RO} whilst maintaining desired functionality and performance.
\subsection{The Multi-Faceted Approach}
Researchers from Iowa State University proposed a multi-faceted approach to trojan detection in \acrshort{FPGA}s~\cite{multiFacetedApproach}.
Their method composed of three approaches:
\begin{itemize}
	\item Functional Testing: A means of feeding test vectors and comparing the output to expected results.
	\item Power Analysis: Using an oscilloscope, the difference in power consumption between the desired design and the modified devices performing the same operations were recorded. Differences were used to discern the presence of a trojan.
	\item Bitfile Analysis: The authors attempted to employ a binary file analysis library named \textit{deBit}~\cite{bitStreamToNetlist} to reverse engineer a netlist from the \gls{Bitstream}.
\end{itemize}
The functional testing method attempted provided reasonable results. 
Test vectors used showed unexpected behavior; this provided only the information that a trojan was present.
The power analysis method again provided results of moderate quality.
With careful placement of the oscilloscope probes the authors were able to infer the physical location on the device where modifications occurred.
This provided no information however as to the relation between the modifications and the design.
Finally, the authors were able to only partially able to convert the \gls{Bitstream} to its netlist description.
Only descriptions of the primary logic circuit elements were achieved.
This could be used to discern some information regarding modifications discovered but is far from creating a complete description of a trojan.


\setlength{\unitlength}{\savedunitlength}
	\startchapter{Automated Trojan Detection}
\label{chapter:trojanDetection}

%%%%%%%%%%%%%%%%%%%%%%%%%%%%%%%
\section{Methodology}
Figure~\ref{fig:Concept} provides a visual representation of the basic concept assumed for the purposes of this work. 
All stages of production of an \acrshort{FPGA} implementation are done "in-house" with the exception of the fabrication process. 
It is assumed that any trojan discovered is inserted in the fabrication phase; all other stages are trusted.  
The method of automated trojan detection described in this work would take place in the 'testing' phase of the life-cycle. 
\begin{figure}[h]
	\centering
	\includegraphics[width=1\linewidth]{figures/Concept}
	\caption[FPGA Life-Cycle]{FPGA Life-Cycle}
	\label{fig:Concept}
\end{figure}

Figure~\ref{fig:methodologyOverview} shows an overview of the trojan detection scheme.
\acrshort{FPGA} designs are written in a \acrfull{HDL}.
\Xilinx provides a series of \acrfull{UI} or command line tools to process the design known as the 'tool-chain'.
The tool chain generates a series of files that are used for a variety of purposes.
The Bit file is a binary representation of the design to be implemented.
It is referred to as the \gls{Bitstream} or Configuration \gls{Bitstream} and is the final form which is loaded into the device.
\begin{figure}
	\centering
	\includegraphics[width=1\linewidth]{Figures/methodologyOverview}
	\caption[Methodology Overview]{Methodology Overview}
	\label{fig:methodologyOverview}
\end{figure}
This Bit file is also one of the primary files sent to the fabrication house where it will be implemented onto the batch of devices ordered.
The resultant files are kept in secure storage while a copy is sent to be fabricated; these 'clean' copies are reffered to as \gls{golden}.
Though it is known that fabrication houses will often attempt to make optimizations on designs this methodology requires that no such efforts are made.
When the completed batch of fabricated chips are returned the \gls{Bitstream} is extracted via the method described in section~\ref{sec:bitstreamExtraction}. 
That which is extracted is referred to as the \gls{target} \gls{Bitstream}.
The \gls{golden} and \gls{target} \gls{Bitstream}s are analyzed in conjunction to detect differences, described in section~\ref{sec:fpgaBitStream}.
Any discovered differences are then attributed to the corresponding component in the architecture, described in section~\ref{sec:tileMapping}.
Finally, descriptive attributes presented in section~\ref{sec:topology} are returned to the user, described in section~\ref{sec:trojanAttributes}. 


%%%%%%%%%%%%%%%%%%%%%%%%%%%%%%%
\section{FPGA Architecture and Configuration} \label{sec:architectureAndConfig}
A \Xilinx \acrfull{FPGA} is comprised of a matrix of blocks referred to as the 'gate-array' and is shown in Figure~\ref{fig:FPGA}.
A device can contain anywhere from a couple hundred to a few thousand blocks; they are arranged into columns by type.
A block will consist of one or multiple tiles depending on the type.
A tile is a component specific to a particular function such as \acrfull{IO}, design logic, memory...etc but their detailed functionality can be configured by the user.
An \acrshort{FPGA} may have over one-hundred different types of tiles however each column is comprised entirely by a single block type.
Columns are separated by clock regions as shown by the dashed lines in Figure~\ref{fig:FPGA}.
Each region is an independent array of blocks that uses a dedicated clock resource; this minimizes clock skew from causing undesired timing delays.
\begin{figure}[h]
	\centering
	\includegraphics[width=1\linewidth]{figures/FPGA}
	\caption[Rudimentary Layout of a Virtex Gate-Array]{Rudimentary Layout of a Virtex Gate-Array}
	\label{fig:FPGA}
\end{figure}
Though each tile has a designated purposes (ex. \acrfull{IO}, \acrfull{CL}, memory...etc) their functionality can be configured by the user; this is how designs are implemented on a device.
The configuration of each tile is dictated by the \gls{Bitstream}. 
To improve performance the contents of the gate-array is referred to as dynamic.
A dynamic device is unable to retain the contents of its memory when it looses power.
To prevent having to plug in a device and download the configuration every time it is powered on, an external static memory device (i.e. retains its contents with loss of power) holds the \gls{Bitstream}. 
When an \acrshort{FPGA} is powered on the \gls{Bitstream} is loaded from the external memory (often \acrfull{SRAM}) into the gate-array, as can be seen in Figure~\ref{fig:architecture}.

\begin{figure}
\centering
\includegraphics[width=0.9\linewidth]{Figures/architecture}
\caption[FPGA Device Layout]{FPGA Device Layout}
\label{fig:architecture}
\end{figure}


%%%%%%%%%%%%%%%%%%%%%%%%%%%%%%%
\section{\gls{Bitstream} Extraction} \label{sec:bitstreamExtraction}
In order to detect any trojans in the \gls{target} device the configuration \gls{Bitstream} will need to be recovered.
As mentioned in section~\ref{sec:architectureAndConfig} the \gls{Bitstream} is stored in a memory unit external to the gate-array.
All \Xilinx devices provide a feature known as \gls{Readback}.
There are two styles of \gls{Readback}; \gls{Readback} verify and \gls{Readback} Capture.
The \gls{Readback} capture method provides a large quantity of debug information which is not needed; \gls{Readback} verify will be used.
\gls{Readback} verify is the process where the device is put into a 'frozen' state during run-time and all of the configuration bits are returned from the gate-array to the \acrshort{SRAM}. 
The results can then be uploaded to a \acrfull{PC} for analysis.
This process overwrites the original frame data in the \acrshort{SRAM} with the values which actually configured the device. 
By using this method rather than simply reading the \acrshort{SRAM} it ensures that what is tested in section~\ref{sec:fpgaBitStream} is actually what configured the device.
This minimizes risk of tampered external memory units or configuration mechanics. 


%%%%%%%%%%%%%%%%%%%%%%%%%%%%%%%
\section{The \acrshort{FPGA} \gls{Bitstream} Analysis} \label{sec:fpgaBitStream}
The \Xilinx \gls{Bitstream} is a binary file composed of a series of 32-bit words organized into 'frames'.
A frame is a string of single bits that span from the top to the bottom of a clock region of a device as seen in the top-right quadrant of Figure~\ref{fig:FPGA}.
A frame affects every block in a column and multiple horizontally adjacent frames are required to configure an entire column.
Each frame is uniquely identified by a 32-bit address and is the smallest addressable element.
The composition of the frame address is fairly consistent across the \Xilinx catalog however there are small differences between device families.
The following is the structure of the Virtex-5 family frame address scheme according to~\cite{virtex5ConfigGuide}.
The make-up of a frame address is shown in Table~\ref{tbl:frameAddress}.
\begin{table}[h]
	\centering
	\caption{Frame Address}
	\label{tbl:frameAddress}
	\resizebox{\textwidth}{!}{
		\begin{tabular}{|c|c|c|c|c|c|c|c|c|c|c|c|c|c|c|c|c|c|c|c|c|c|c|c|c|c|c|c|c|c|c|c|}
			\hline
			\multicolumn{8}{|c|}{Unused} & \multicolumn{3}{c|}{BA} & T & \multicolumn{5}{c|}{Row Address} & \multicolumn{8}{c|}{Major Address} & \multicolumn{7}{c|}{Minor Address} \\ \hline
			31 & 30 & 29 & 28 & 27 & 26 & 25 & 24 & 23 & 22 & 21 & 20 & 19 & 18 & 17 & 16 & 15 & 14 & 13 & 12 & 11 & 10 & 9 & 8 & 7 & 6 & 5 & 4 & 3 & 2 & 1 & 0 \\ \hline
			0 & 0 & 0 & 0 & 0 & x & x & x & x & x & x & x & x & x & x & x & x & x & x & x & x & x & x & 0 & 0 & 0 & 0 & 0 & 0 & 0 & 0 & 0 \\ \hline
		\end{tabular}		
	}
\end{table}
The \acrfull{BA} identifies the block type.
\begin{itemize}
	\item BA 0: Logic type.
	\item BA 1: \acrfull{BRAM}.
	\item BA 2: \acrshort{BRAM} Interconnect.
	\item BA 3: \acrshort{BRAM} non-configuration frame.
\end{itemize}
The logic block contains the columns which provides the primary configuration for the device (\acrshort{CLBs}, \acrshort{IOBs}... etc).
The \acrshort{BRAM} columns initialize the memory for the device while the \acrshort{BRAM} Interconnect columns configure how the logic of the design interacts with the \acrshort{BRAM}.

In the case of the Virtex-5 family each clock region is composed of twenty blocks in a column separated by a horizontal clock bus as shown in Figure~\ref{fig:RowOrder}.
\begin{figure}[]
\centering
\includegraphics[width=.7\linewidth]{Figures/RowOrder}
\caption[Row Order of Virtex-5 Clock Region]{Row Order of Virtex-5 Clock Region}
\label{fig:RowOrder}
\end{figure}
Each row of blocks is given a row value in it's address that increments away from the clock bus starting at 0. 
The frame address includes a Top indicator bit in position 20 that indicates whether the specified row is above or below the horizontal clock bus.
The major address specifies the block within the row.
These addresses are numbered from left to right and begin at 0.
The minor address indicates the frame number within a column. 
Table~\ref{tbl:minorAddressNumbers} provides the number of frames per column type.
\begin{table}[]
	\centering
	\caption{Number of Frames (minor addresses) per Column}
	\label{tbl:minorAddressNumbers}
	\begin{tabular}{|c|c|}
		\hline
		Block             & Number Of Frames \\ \hline
		CLB               & 36               \\ \hline
		DSP               & 28               \\ \hline
		\acrshort{BRAM}   & 30               \\ \hline
		IOB               & 54               \\ \hline
		Clock             & 4                \\ \hline
	\end{tabular}
\end{table}
As described in section~\ref{sec:architectureAndConfig} a block may contain multiple tiles.
In a \acrshort{CLB} column a block consists of an interconnect tile, also known as a \acrfull{SM} and a \acrshort{CLB}.
Frames are numbered from left to right, starting with 0. 
For each block, except in a clock column, frames numbered 0 to 25 access the Interconnect for that column. 
For all blocks, except the \acrshort{CLB} and the clock column, frames numbered 26 and 27 access the Interface for that column. 
All other frames are specific to that block.~\cite{virtex5ConfigGuide}
To further understand how frames configure tiles a mapping must be made between each frame and the corresponding tile.
This is described in section~\ref{sec:tileMapping}.

%%%%%%%%%%%%%%%%%%%%%%%%%%%%%%%
\section{Tile Mapping} \label{sec:tileMapping}
The process of tile mapping aims to attribute each word in a configuration frame to the component of the device it configures.
In the case of Virtex-5 devices a frame is composed of 41 words that can be thought of as a vertical stack that aligns with a column.
As described in section~\ref{sec:fpgaBitStream} a row consists of a stack of basic blocks; there are 20 \acrshort{CLB} blocks per column, 40 \acrshort{IOB}s, 4 \acrshort{BRAM}...etc.
\begin{figure}[h]
	\centering
	\includegraphics[width=0.9\linewidth]{Figures/frameTileMap}
	\caption[Configuration Words in the Bitstream~\cite{virtex5ConfigGuide}]{Configuration Words in the Bitstream~\cite{virtex5ConfigGuide}}
	\label{fig:frameTileMap}
\end{figure}
As can be seen in Figure~\ref{fig:frameTileMap} the central word in a frame configures the horizontally running clock bus.
The remaining words are used to configure the blocks in the column.
The purpose of the central word in the column is known to be mapped to the clock bus.
For the purposes of the following computations it is considered removed from the frame.
From this, equation~\ref{eqn:numWordsPerBlock} can be deduced which is used to compute the number of 32-bit words that span each block.
\begin{equation} \label{eqn:numWordsPerBlock}
n = (W - C) + B
\end{equation}
where:
\begin{conditions}
	n     &  Number of Words per Block \\
	W     &  Number of 32-bit words per frame \\   
	C     &  Number of clock words per frame \\
	B     &  Number of blocks per column
\end{conditions}
As shown in Figure~\ref{fig:frameTileMap} words are addressed from the 'top' of a device down.
Equation~\ref{eqn:getTileNumber} can be used to map a particular word in a frame to a tile on the device.
\begin{equation} \label{eqn:getTileNumber}
i = B - \floor*{\frac{w}{n}}
\end{equation}
where:
\begin{conditions}
	i     &  Word Number in frame\\
	B     &  Number of blocks per column \\
	w     &  Word number \\
	n     &  Number of Words per Block 
\end{conditions}
With equations~\ref{eqn:numWordsPerBlock} and~\ref{eqn:getTileNumber} it is now possible attribute any modifications in the \gls{Bitstream} to its corresponding component.
The \gls{golden} and \gls{target} \gls{Bitstream}s undergo a comparison process.
Any frames that contain words that have been modified are stored for analysis.
The address of a modified frame is used to determine the row and column as described in section~\ref{sec:fpgaBitStream}.
\subsection{Column-SubColumn Mapping}
\subsection{SubColumn-Frame Mapping}
\subsection{Tile-Word Mapping}

\section{Determining Trojan Attributes} \label{sec:trojanAttributes}
The purpose of attributes recap.
\subsection{Direct Extraction Methods}
\subsubsection{Observed Location Attributes}
The presence of attributes in the \textit{Location} category are directly observable from the results of the tile mapping method described in section~\ref{sec:tileMapping}.
\Xilinx tiles conform to purpose-specific groups or block types which were discussed in section~\ref{sec:fpgaBitStream}.
These block types contain sub-types that perform actions which pertain to the \textit{Location}, category. 
\begin{enumerate}
	\item The \textbf{Processor} attribute pertains to the core functionality of the design logic. It can be awarded for presence of a modified \acrshort{CLB} tile or Interconnect tile.
	\item The \textbf{Memory} attribute can be awarded for the presence of modified \acrshort{BRAM} components.
	\item The \textbf{\acrshort{IO}} attribute can be awarded for presence of modified \acrshort{IOB} tiles.
	\item The \textbf{Power Supply} attribute can be awarded for the presence of modified interface or configuration tiles.
	\item The \textbf{Clock Grid} attribute can be awarded for modified clock tiles.
\end{enumerate}
\subsubsection{Scatter Score Method}
\subsubsection{Insertion and Abstraction Attributes}
\subsection{Relation Matrix Use}


	\startchapter{Software Implementation}
\label{chapter:implementation}
\section{Introduction}
The \Name was build to be a stand-alone application.
In order to allow it to be a powerful yet simple application the technologies used were selected with the idea of portability in mind.

\section{Technologies Used}
\subsection{\Xilinx}
\Xilinx is one of the two largest manufacturers of \acrshort{FPGA}s. 
Their devices are considerably more popular than their competitors.
The configuration of devices employs a well known series of steps referred to as the \Xilinx 'tool-chain'~\cite{xilnxDevManual}.
The 'tool-chain' not only compiles user designs and constraints into the configuration \gls{Bitstream} but performs a series of complex operations to optimize designs and effectively implement them on any \Xilinx model of the user's choosing.
\begin{enumerate}
	\item \textbf{\gls{sch2hdl}}: \glsdesc{sch2hdl}
	\item \textbf{\gls{XST}}: \glsdesc{XST}
	\item \textbf{\gls{MAP}}: \glsdesc{MAP}
	\item \textbf{\gls{PAR}}: \glsdesc{PAR}
	\item \textbf{\gls{ngdbuild}}: \glsdesc{ngdbuild}
	\item \textbf{\gls{trce}}: \glsdesc{trce}
	\item \textbf{\gls{Bitgen}}: \glsdesc{Bitgen}
\end{enumerate}
Each step produces a series of files that are used for a variety of purposes. 
Often the resultant files are used by the subsequent step but some are intended for user information.
The \gls{ngdbuild} tool generates what is known as the netlist for the design.
The netlist is a description of the connectivity of the circuit implemented on the device. 
The generated netlist is in a non human-readable format in an 'ndc' file.
Fortunately, \Xilinx provides an additional tool called \acrshort{xdl2ndc} which allows the conversion to the human-readable \acrfull{XDL}.
\subsubsection{\acrfull{XDL}}
The \acrfull{XDL} is a human-readable ASCII format; though it is not actively part of the 'tool-chain' it is considered a native netlist format for describing and representing \acrshort{FPGA} designs~\cite{xdlTutorial}. 
A part is a human-defined component which is to be implemented.
Netlists either contain or refer to descriptions of the parts used and where in the device they are implemented.
When a part is used it is called an 'instance'; thus each 'instance' has a definition, sometimes referred to as a master.

An \acrshort{XDL} file contains two sections, the instance placement and configuration section, and the net routing section. 
The placement and configuration section provides a list of every instance in a design. 
Their descriptions include all of the configuration details required for their implementation on a component (power settings, logic, timing configurations... etc).
In the \Xilinx jargon, a net refers to any electrical path between two components; more specifically, a net describes a communication channel.
The gate-array of a \Xilinx device is composed of a grid of wires.
These wires can be fused together by \acrfull{PIP} to make a useful connection between two components, thus creating a net.
The output-pin of a net receives the signal from the transmitting component while the input-pin delivers the signal to the receiver.
The \acrshort{PIP}s in-between dictate the path the signal takes between the two components.
The net routing section of the \acrshort{XDL} file describes every path in the design.
Combined, these two sections completely describe a design and how it is implemented.
\subsubsection{XDLRC} \label{sec:XDLRC}
The \acrshort{XDL} file describes the design implemented on a device.
From this a lot of information regarding the composition of a \Xilinx device can be learned but it does not provide the entire description; only what has been used by the design.
The \acrshort{xdl2ndc} command line tool provides an option to generate a resource report file referred to as an XDLRC file. 
A XDLRC file describes the entire architecture of a \Xilinx \acrshort{FPGA}.
\ConditionSize
\begin{lstlisting}[label={lst:xdlrc}, language=Python, caption={A hierarchical XDLRC resource description of a Spartan 6 FPGA consisting of a header, a tile section, and a trailing device summary~\cite{xdlTutorial}}]
#Header section
(xdl_resource_report v0.2 xc6slx16csg324-3 spartan6
# Device Level Dimensions
(tiles 73 62
...
	#Configurable logic block with two slices
	(tile 4 6 CLEXL_X1Y61 CLEXL 2
		(primitive_site SLICE_X0Y61 SLICEL internal 45
			(pinwire A1 input L_A1)
		...
		(primitive_site SLICE_X1Y61 SLICEX internal 43
		...
		(pinwire D output XX_D)
	...)
	# Interconnect tile
	(tile 4 5 INT_X1Y61 INT 1
	...
		(wire EE2B0 2
			(conn CLEXM_X2Y61 CLEXM_EE2M0)
			(conn INT_BRAM_X3Y61 EE2E0)
		...
		# Programmable Interconnect Points
		(pip INT_X1Y61 EE2E0 -> EE2B0)
		(pip INT_X1Y61 EE4E0 -> EE2B0)
		(pip INT_X1Y61 EL1E_S0 -> LOGICIN_B9)
	...)
# summary
(summary tiles=4526 sites=5378 sitedefs=46 numpins=157962 numpips=5782505))
\end{lstlisting}
\normalsize

Listing~\ref{lst:xdlrc} provides and example of the XDLRC format.
The report begins with a header describing the device.
It then reports that the overall architecture contains a matrix of tiles 73-wide and 62-tall.
Further down we can see a configurable logic block at coordinate (4,6) and an interconnect tile at coordinate (4,5) are shown.
Each of these tile descriptions provide details of the subcomponents they contain, pinwires, \acrshort{PIP}s, slices...etc.
The tool is capable of generating a variety of different XDLRC files for a device, ranging in the level of detail.
The smaller files can range around 10MB while the fully detailed descriptions can reach 7GB.
\subsection{Java} \label{sec:java}
Java is a powerful and general-purpose programming language. 
It is specifically designed to be as independent as possible.
The original developer, James Gosling, intended the language to allow developers a comfortable implementation experience with seamless deployment. 
The custodians of the Java language, Sun Microsystems, promotes the slogan 'write-once, run anywhere'.  
\Name was written in Java primarily in order to interface with the \acrshort{API} known as RapidSmith which is described in section~\ref{sec:rapidSmith}.
However, the additional benefits of allowing \Name to be compact, cross-platform application can not be understated.
The Java language provides a native \acrfull{GUI} toolkit known as \Swing.
\Swing is an \acrshort{API} that is part of Oracle's Java Foundation Classes; in other words it is readily available to all users of Java.
It provides a simple to use programming structure for creating sophisticated \acrshort{UIs}.
\Name employs Java and Java \Swing to make it as user-friendly as possible. 

\subsection{RapidSmith} \label{sec:rapidSmith}
RapidSmith is a set of tools and \acrfull{APIs} written in Java that enable \acrfull{CAD} tool creation for \Xilinx \acrshort{FPGA}s~\cite{rapidSmith}.
Its purpose is to be used as a rapid prototyping platform for experimentation and research.
The code is freely to use readily accessible.
It was chosen as a supporting library for \Name for several reasons.
First, the code base provides a series of class structures that astutely mirrors structure of \Xilinx devices and infrastructure.
Secondly, it provides ready-made tools for extracting frames \gls{Bitstream} files. 
\gls{Bitstream} files are huge binary files, without the tools provided by RapidSmith the analysis of these files becomes an arduous task.
Finally, and most importantly, the creators of RapidSmith have developed a means of condensing XDLRC files into a greatly compressed format referred to as a 'database' file.
\Name requires considerable detail of an \acrshort{FPGA}'s architecture in order to accurately described the effects a trojan has on a design.
As mentioned in section~\ref{sec:XDLRC}, the fully detailed XDLRC files can reach 7GB in size.
Working with text-based files of this size determents performance to an unusable level.
\subsubsection{Class Structure} \label{sec:classStructure}
\begin{figure*}[h]
	\centering
	\begin{subfigure}[t]{0.5\textwidth} \label{fig:rapidSmithDesign}
		\centering
		\includegraphics[height=2.1in]{Figures/rapidSmithDesign}
		\caption{The Class Heirarchy For the RapidSmith Design Object}
	\end{subfigure}%
	~ 
	\begin{subfigure}[t]{0.5\textwidth} \label{fig:rapidSmithDevice}
		\centering
		\includegraphics[height=2.1in]{Figures/rapidSmithDevice}
		\caption{The Class Heirarchy For the RapidSmith Device Object}
	\end{subfigure}
	\caption{The RapidSmith Class Heirarchy~\cite{rapidSmithManual}}
\end{figure*}
\subsubsection{Database Files} \label{sec:databaseFiles}
\section{Design}
\section{\acrfull{UI}}
	\include{chapters/5/chapter_Website}
	\startchapter{Contributions}
\label{concl}
%\begin{itemize}
%	\item \textbf{How was the hypothesis been answered?}: The analysis performed on the three benchmarks effectively demonstrate this new means of trojan detection in \acrshort{FPGA}s. 
%	\item \textbf{How has the field changed with this knowledge?}: without this FPGA IC manufactureres lack a viable means of detection.
%\end{itemize}
Configuration \gls{Bitstream}s are enormous strings of binary data.
To the human reader this information means nothing.
To an \acrshort{FPGA}, however, this data is everything.
Every conceivable design, and every possible trojan is contained within the \gls{Bitstream}.
Yet, due to the shear volume of information within it and the lack of details on its format it has not previously been a common subject of study.

The purpose of this work is to provide an easy-to-use application that automates the processes of discovering and analyzing malicious modifications.
With its success, \acrlong{IC} manufacturers that use \acrshort{FPGA}s will have an additional tool to ensure that their products operate as expected.
Using the \NameNoPeriod takes only a few button clicks on the \acrlong{UI}.
Its simple construction does not require any additional software or complicated install procedures and it can be used on any major operating system.
Ensuring chips that have returned from Fabrication operate as expected takes no more than a few minutes.
With the use of the \NameNoPeriod manufacturers will not need to train employees, buy expensive equipment or waste man-hours on additional testing.

The contributions of this work are:
\begin{enumerate}
	\item Developed a method of extracting all of the information necessary to detect and analyze trojans directly from the configuration \gls{Bitstream}. 
	\item Described methods for directly observing the presence of attributes in detected trojans.
	\item Provided a method for inferring attributes that are not directly observable using a known taxonomy.
	\item Developed a user-friendly application which automates the described methodology.
	\item Succeeded in ensuring the application is cross-platform and does not require a complicated install procedure.
\end{enumerate}




%My first rule for this chapter is to avoid finishing it with a section talking about future work. It may seem logical, yet it also appears to give a list of all items which remain undone! It is not the best way psychologically.
%
%This chapter should contain a mirror of the introduction, where a summary of the \textit{extraordinary} new results and their wonderful attributes should be stated first, followed by an executive summary of how this new solution was arrived at. Consider the practical fact that this chapter will be read quickly at the beginning of a review (thus it needs to provide a strong impact) and then again in depth at the very end, perhaps a few days after the details of the previous 3 chapters have been somehow forgotten. Reinforcement of the positive is the key strategy here, without of course blowing hot air.
%
%One other consideration is that some people like to join the chapter containing the analysis with the only with conclusions. This can indeed work very well in certain topics.
%
%Finally, the conclusions do not appear only in this chapter. This sample mini thesis lacks a feature which I regard as absolutely necessary, namely a short paragraph at the end of each chapter giving a brief summary of what was presented together with a one sentence preview as to what might expect the connection to be with the next chapter(s). You are writing a story, the \textit{story of your wonderful research work}. A story needs a line connecting all its parts and you are responsible for these linkages.

	\appendix
	\include{chapters/appendix/chapter_app}

% The style of bibliography exemplified here is the "plain",
% normally used in science theses. This is shown
% by the entry {plain} below. Substitute the
% appropriate bibliography style. See also the
% PDF file "InformationOnBibliographyStyles" in this
% directory for more choices.

% The Bibliography file is a BibTex file named
% UVicThesis.bib and called below

	\TOCadd{Bibliography}
	\bibliographystyle{plain}
	%\bibliographystyle{./IEEEtran}
	\bibliography{UvicThesis}

\end{document}
